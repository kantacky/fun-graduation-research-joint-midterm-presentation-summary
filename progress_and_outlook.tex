\section{進捗と計画}
\subsection{卒業研究でのゴール}
\ref{section_method}章で述べた手法では,対象を京都エリアの一部地域に限定するなど,シミュレーションの簡略化のための制限がいくつかある.その制限のもとで,ある程度シミュレーションを簡略化し,エージェントの意思決定機構を設計し,シミュレーションを行う.

\subsection{進捗状況と計画}
現在の進捗状況は以下の通りである.
\begin{itemize}
  \item \ref{section_agent}節で述べたエージェントの定義を確立
  \item 公共交通機関の運行スケジュール収集
\end{itemize}

今後の計画は以下の通りである.
\begin{itemize}
  \item モバイル空間統計を用いた移動パターンの分析
  \item Google Maps Platformを用いたランドマークの抽出
  \item エージェントの意思決定機構の構築
  \item ルート検索システムの構築 (既存のルート検索システムの利用を予定)
  \item シミュレーションの実行と結果の分析
\end{itemize}

\subsection{展望}
卒業研究でのゴールを達成した後には,エージェントに与える情報を操作した場合や,エージェントが他の交通機関を利用する場合を考慮した,追加のシミュレーションを行うことを考えている.
