\section{進捗と計画}
\subsection{卒業研究でのゴール}
\ref{section_method}章で述べた手法では,フィールドを京都エリアの一部地域に限定すること,エージェントの公共交通機関乗降スポットとランドマーク間の移動を考慮しないことなど,シミュレーションの簡略化のための制限がいくつかある.その制限のもとで,ある程度シミュレーションを簡略化し,エージェントの目的地決定アルゴリズムとルート選択アルゴリズムを設計し,シミュレーションを行う.

\subsection{進捗状況と計画}
現在の進捗状況は以下の通りである.
\begin{itemize}
  \item \ref{section_agent}節で述べたエージェントの定義を確立
  \item 公共交通機関の運行スケジュール収集
\end{itemize}

今後の計画は以下の通りである.
\begin{itemize}
  \item モバイル空間統計を用いた移動パターンの分析
  \item Google Maps Platformを用いたランドマークの抽出
  \item エージェントの判断アルゴリズムの構築
  \item ルート検索アルゴリズムの構築 (既存のルート検索システムの利用を予定)
  \item シミュレーションの実行と結果の分析
\end{itemize}

\subsection{展望}
卒業研究でのゴールを達成した後には,エージェントに与える混雑度などの情報を意図的に操作したり,エージェントのセンサ,判断アルゴリズムを変更するなどして,追加のシミュレーションを行うことを考えている.また,評価手法についても検討する.
