\section{はじめに}
国土交通省によると,観光客が集中する一部地域では,公共交通機関の混雑や交通渋滞などの問題により,地元住民の生活への影響,旅行者の満足度低下への懸念が生じている\cite{kokudo2024}.2024年3月には,課題分析に基づく具体的な対策計画の策定,取組の実施について支援していく「先駆モデル地域」として全国20地域が選定された.その中でも,京都エリアでは,公共交通等の混雑対策地域としてさまざまな取り組みが行われている.

本研究では,京都エリアを対象とした公共交通機関利用者のマルチエージェントシミュレーションを行い,公共交通機関の混雑度を可視化することを目的とする.
