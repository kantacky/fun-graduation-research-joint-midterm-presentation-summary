\section{関連研究}
\subsection{マルチエージェント交通シミュレーションへの法的推論機構の導入}
吉添らは,マルチエージェント交通シミュレーションにおいて,車両エージェントの行動決定の仕組みについて,シミュレーション外部で定義した交通法規・施策に準じて法的推論処理を行うシステムとの連携機構の実装を行なった\cite{yoshizoe2023}.この研究では,モデリング・シミュレーション基盤であるGAMA\footnote{https://gama-platform.org}をプラットフォームとして構築したMACiMAという独自マルチエージェント都市シミュレータを利用している.

吉添らは,車両を対象にしたシミュレーションを行っているのに対し,本研究では公共交通機関利用者を対象としている.また,吉添らは,シミュレーション外部で定義した交通法規・施策に準じて法的推論処理を行うシステムとの連携機構の実装を行なっているが,本研究では,公共交通機関利用者の移動パターンを分析し,その傾向からエージェントの目的地決定機構を設計する.

本研究では,観光地における公共交通機関利用者のマルチエージェントシミュレーションを行うプラットフォームとして,GAMAを利用する.シミュレータ外部のシステムとの連携機構について,吉添らの研究を参考にする.
